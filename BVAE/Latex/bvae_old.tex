

\documentclass{article}

\usepackage{graphicx}
\usepackage{amsmath}
\usepackage{tikz}
\usetikzlibrary{bayesnet}

\begin{document}

\title{Generator-Aware Bayesian Variational Autoencoders}

\author{Chris}

\maketitle

% \begin{abstract}
% In some scenarios, it is important to have uncerntainty in the generator. However, naively usign a BNN as the generator will be limited by (mutual info argument). Here we show how to build a GABVAE. 
% \end{abstract}



% \section{Introduction}

% VAEs overfit in both the recogntion and generator networks. The recogniton overfiting is solvable. The generator overfittign gets worse with higher latent dimensionality Q and input dimensionality D. 

\begin{figure}[h]
    \centering
    \tikz
    {
        \node[latent] (weights) {$\theta$} ;
        \node[latent, right=of weights] (Z) {$Z$} ;
        \node[latent, left=of weights] (Z_prime) {$Z'$} ;
        \node[obs, below=of Z] (X) {$X$} ;
        \node[obs, below=of Z_prime] (X_prime) {$X'$} ;
        
        \edge {Z_prime} {X_prime} ;
        \edge {weights} {X_prime} ;
        \edge {weights} {X} ;
        \edge {Z} {X} ;
        
        \plate[inner sep=0.25cm, xshift=0.0cm, yshift=0.12cm] {plate1} {(Z) (X)} {N};
    }
    \caption{Graphical model of BVAE. $X$ is the training set. $X'$ is the test set.}
\end{figure}





\section{BVAE Objectives}

\subsection{Objective maximized during training}



\begin{align}
    p(x) &= \int_{\theta} \int_{z} p(x,z,\theta) \\
    &= E_{q(\theta,z|x)} \left[ \frac{p(x,z,\theta)}{q(\theta,z|x)} \right ] \\
    &= E_{q(\theta,z|x)} \left[ \frac{p(x,z)p(\theta)}{q(z|x,\theta)q(\theta)} \right ] \\
    log(p(x)) &= log \left( E_{q(\theta,z|x)} \left[ \frac{p(x,z)p(\theta)}{q(z|x,\theta)q(\theta)} \right ] \right) \\
    &\geq  E_{q(\theta,z|x)} \left[ log \left( \frac{p(x,z)p(\theta)}{q(z|x,\theta)q(\theta)} \right) \right ] 
\end{align}


\subsection{Objective evaluated on the test set}


\begin{align}
    p(x'|x) &= \int_{\theta} \int_{z'} p(x',z',\theta|x) \\
    &= \int_{\theta} \int_{z} p(x',z'|\theta,x) p(\theta|x) \\
    &= \int_{\theta} \int_{z} p(x',z'|\theta) p(\theta|x) \\
    q(x'|x) &= \int_{\theta} \int_{z} p(x',z'|\theta) q(\theta) \\
    &= E_{q(\theta)} \left[ \int_{z} p(x',z'|\theta) \right] \\
    &= E_{q(\theta)} \left[ E_{q(z'|x')} \left[ \frac{p(x',z'|\theta)}{q(z'|x',\theta)} \right] \right]  \\
	log(q(x'|x)) &= log \left(  E_{q(\theta)} \left[ E_{q(z'|x')} \left[ \frac{p(x',z'|\theta)}{q(z'|x',\theta)}  \right ] \right ] \right) \\
    &\geq   E_{q(\theta)} \left[ E_{q(z'|x')} \left[ log \left( \frac{p(x',z'|\theta)}{q(z'|x',\theta)} \right) \right ] \right ] 
\end{align}



\subsection{Other}


\begin{align}
    log p(x) = log \int p(x|\theta) q(\theta) = log E_q \left[ p(x|\theta) \right]
\end{align}



    % &= E_{q(\theta,z|x)} \left[ \frac{p(x,z,\theta)}{q(\theta,z|x)} \right ] \\
    % &= E_{q(\theta,z|x)} \left[ \frac{p(x,z)p(\theta)}{q(z|x)q(\theta)} \right ] \\
    % log(p(x)) &= log \left( E_{q(\theta,z|x)} \left[ \frac{p(x,z)p(\theta)}{q(z|x)q(\theta)} \right ] \right) \\

\section{2D Plots}

Equation used to plot $q_{T}(z_{T})$

\begin{align}
    q_{T}(z_{T}) &= E_{q(v_T)} \left[ \frac {q_{0}(T_{1}^{-1}(...(T_{T}^{-1}(z_{T},v_{T}))} { q(v_{T})}  \right ]
\end{align}

\end{document}



















% \pdfoutput=1
% \documentclass{article} % For LaTeX2e
% % \usepackage{include/nips15submit_e}
% % \usepackage{include/nips_2016}
% % \usepackage{include/nips15submit_e,times}

% \usepackage{hyperref}
% \usepackage{url}
% %\documentstyle[nips14submit_09,times,art10]{article} % For LaTeX 2.09
% \usepackage{graphicx}
% \usepackage{bm}
% \usepackage{breqn}
% \usepackage{amssymb,amsmath}%,amsthm}

% \title{HIS+BVAE+more}


% % \author{
% % Chris Cremer \\
% % University of Toronto\\
% % \texttt{ccremer@cs.toronto.edu} \\
% % \and
% % David Duvenaud \\
% % University of Toronto \\
% % \texttt{duvenaud@cs.toronto.edu} \\
% % }

% % \newcommand{\fix}{\marginpar{FIX}}
% % \newcommand{\new}{\marginpar{NEW}}

% \nipsfinalcopy % Uncomment for camera-ready version

% \begin{document}

% \maketitle

% % \begin{abstract}
% % TODO

% % \end{abstract}

% % \section{Introduction}
% % TODO

% \section{2D Plots}

% \begin{align}
%     q_{T}(z_{T}) &= E_{q(v_T)} \left[ \frac {q_{0}(T_{1}^{-1}...T_{T}^{-1}(z_{T},v_{T}))} { q(v_{T})}  \right ]
% \end{align}

% % \section{Method}
% % TODO 

% % \section{Related Work}
% % TODO

% % \section{Experimental Results}
% % TODO

% % \section{Conclusion}
% % TODO


% \bibliographystyle{ieeetr}
% \bibliography{include/bvae.bib}

% \end{document}
